\section{Professional experience}\label{professional-experience}

\begin{description}
\tightlist
\item[2021--23]
Research Scientist, Stock Assessment and Research Division, Pacific
Biological Station, Fisheries and Oceans Canada, Nanaimo, BC, Canada
\item[2019--23]
Associate Editor, the Canadian Journal of Fisheries and Aquatic Sciences
\item[2020--23]
Adjunct Professor, Department of Mathematics, Simon Fraser University
\item[2017--20]
Program Head in Analytical Approaches and Spatial Stock Assessment,
Groundfish Section, Stock Assessment and Research Division, Pacific
Biological Station, Fisheries and Oceans Canada, Nanaimo, BC, Canada
\item[2015--17]
David H. Smith Conservation Research Fellow, School of Aquatic and
Fishery Sciences, University of Washington and Northwest Fisheries
Science Center, Seattle, WA
\item[2015]
Postdoctoral Fellow, School of Resource and Environmental Management,
Simon Fraser University
\item[2011--15]
Graduate Research Assistant, Department of Biological Sciences, Simon
Fraser University
\item[2008--11]
Graduate Research Assistant, Department of Biology, Dalhousie University
\end{description}

\section{Education}\label{education}

\begin{description}
\tightlist
\item[2011--15]
Ph.D.\ Biology, Simon Fraser University, Burnaby, Canada\\
Variance and extreme events in population ecology\\
Fulbright Scholar to the University of Washington in 2012--13\\
Dean of Graduate Studies Convocation Medal
\item[2008--10]
M.Sc.\ Biology, Dalhousie University, Halifax, Canada\\
Trends, drivers, and ecosystem effects of expanding global invertebrate
fisheries\\
Canadian Governor General's Academic Gold Medal
\item[2004--07]
B.Sc.\ (Hons.) Environmental Science, Dalhousie University, Halifax,
Canada\\
First Class Honours \& Environmental Programmes Honour Society Medal
\item[2001--03]
B.Sc.\ Candidate (Hons.) Kinesiology, University of Waterloo, Waterloo,
Canada; transferred to Dalhousie University in 2004.
\end{description}

\section{Preprints}\label{preprints}

\begin{description}
\tightlist
\item[2024]
Thorson, J.T., \textbf{S.C. Anderson}, P. Goddard, C.N. Rooper.
tinyVAST: R package with an expressive interface to specify lagged and
simultaneous effects in multivariate spatio-temporal models. arXiv.
\url{https://doi.org/10.48550/arXiv.2401.10193}.
\item[2024]
\textbf{Anderson, S.C.}, E.J. Ward, P.A. English, L.A.K. Barnett, J.T.
Thorson. sdmTMB: an R package for fast, flexible, and user-friendly
generalized linear mixed effects models with spatial and spatiotemporal
random fields. bioRxiv 2022.03.24.485545.
\url{https://doi.org/10.1101/2022.03.24.485545}.
\item[2021]
\textbf{Anderson, S.C}, B.M. Connors, P.A. English, R.E. Forrest, R.
Haigh, K.R. Holt. Trends in Pacific Canadian groundfish stock status.
bioRxiv 2021.12.13.472502.
\url{https://doi.org/10.1101/2021.12.13.472502}.
\end{description}

\section{Publications}\label{publications}

\begin{description}
\tightlist
\item[2024]
Freshwater, C., \textbf{S.C. Anderson}, D.D. Huff, J.M. Smith, D.
Jackson, B. Hendriks, S.G. Hinch, S. Johnston, A.W. Trites, and J. King.
Chinook salmon depth distributions on the continental shelf are shaped
by interactions between location, season, and individual condition.
Movement Ecology 12(1): 21.
\url{https://doi.org/10.1186/s40462-024-00464-y}.
\item[2024]
\textbf{Anderson, S.C.}, P.A. English, K.S.P. Gale, D.R. Haggarty, C.K.
Robb, E.M. Rubidge, P.L. Thompson. Impacts on population indices if
scientific surveys are excluded from marine protected areas. ICES
Journal of Marine Science. fsae009.
\url{https://doi.org/10.1093/icesjms/fsae009}.
\item[2024]
Hughes, B.B., Beheshti, K.M., Tinker, M.T., Angelini, C., Endris, C.,
Murai, L., \textbf{Anderson, S.C.}, Espinosa, S., Staedler, M.,
Tomoleoni, J.A., Sanchez, M., and Silliman, B.R. Top-predator recovery
abates geomorphic decline of a coastal ecosystem. Nature 626(7997):
111--118. \url{https://doi.org/10.1038/s41586-023-06959-9}.
\item[2024]
English, P.A., C.M. Picco, J.C. Edwards, D.R. Haggarty, R.E. Forrest,
\textbf{S.C. Anderson}. Spatial restrictions hinder avoidance of choke
species in an Indigenous rights-based fishery. People and Nature. 6:
75--90. \url{https://doi.org/10.1002/pan3.10554}.
\item[2023]
Nephin, J., P.L. Thompson, \textbf{S.C. Anderson}, A.E. Park, C.N.
Rooper, B. Aulthouse, and J. Watson. Integrating disparate survey data
in species distribution models demonstrate the need for robust model
evaluation. Can. J. Fish. Aquat. Sci. 80(12): 1869-1889.
\url{https://doi.org/10.1139/cjfas-2022-0279}.
\item[2024]
Baker, K.D., \textbf{S.C. Anderson}, D.R.J. Mullowney, W. Walkusz, and
K.R. Skanes. Moving away from a scale mismatch: Spatiotemporal modelling
of striped shrimp (\emph{Pandalus montagui}) density in Canada's
subarctic. Fisheries Research. 270: 106898.
\url{https://doi.org/10.1016/j.fishres.2023.106898}.
\item[2023]
Yalcin, S., \textbf{S.C. Anderson}, P.M. Regular, P.A. English.
Exploring the limits of spatiotemporal and design-based index
standardization under reduced survey coverage. ICES Journal of Marine
Science. 80(9): 2368--2379.
\url{https://doi.org/10.1093/icesjms/fsad155}.
\item[2023]
K.L. Blake, D. Veríssimo, \textbf{S.C. Anderson}, A. Gleave. Do species
receive more attention when named after celebrities? Conservation
Biology. e14184. \url{https://doi.org/10.1111/cobi.14184}.
\item[2023]
Davies S.C., P.L. Thompson, C. Gomez, J. Nephin, A. Knudby, A.E. Park,
S.K. Friesen, L.J. Pollock, E.M. Rubidge, \textbf{S.C. Anderson}, J.C.
Iacarella, D.A. Lyons, A.A. MacDonald, A. McMillan, E.J. Ward, A.M.
Holdsworth, N. Swart, J. Price, K.L. Hunter. Addressing uncertainty when
projecting marine species' distributions under climate change.
Ecography: e06731. \url{https://doi.org/10.1111/ecog.06731}.
\item[2023]
Liu, O.R. E.J. Ward, \textbf{S.C. Anderson}, K.S. Andrews, L.A.K.
Barnett, S. Brodie, G. Carroll, J. Fiechter, M.A.\ Haltuch, C.J. Harvey,
E.L. Hazen, P.Y. Hernvann, M. Jacox, I.C. Kaplan, S. Matson, K. Norman,
M.P. Buil, R.L. Selden, A.O. Shelton, J.F. Samhouri. Species
redistribution creates unequal outcomes for multispecies fisheries under
projected climate change. Science Advances 9(33): eadg5468.
\url{https://doi.org/10.1126/sciadv.adg5468}.
\item[2023]
Lindmark, M., \textbf{S.C. Anderson}, M. Gogina, M. Casini. Evaluating
drivers of spatiotemporal variability in individual condition of a
bottom-associated marine fish, Atlantic cod (\emph{Gadus morhua}). ICES
Journal of Marine Science. 80(5): 1539--1550.
\url{https://doi.org/10.1093/icesjms/fsad084}.
\item[2023]
Free, C.M., \textbf{S.C. Anderson}, E.A. Hellmers, B.A. Muhling, M.O.
Navarro, K. Richerson, L.A. Rogers, W.H. Satterthwaite, A.R. Thompson,
J.M. Burt, S.D. Gaines, K.N. Marshall, J.W. White, and L.F. Bellquist.
Impact of the 2014--2016 marine heatwave on US and Canada West Coast
fisheries: Surprises and lessons from key case studies. Fish and
Fisheries. 24: 652--674. \url{https://doi.org/10.1111/faf.12753}.
\item[2023]
Thompson, P.L., \textbf{S.C. Anderson}, J. Nephin, C.K. Robb, B.
Proudfoot, A.E. Park, D.R. Haggarty, E.M. Rubidge. Integrating trawl and
longline surveys across British Columbia improves groundfish
distribution predictions. Can. J. Fish. Aquat. Sci. 80(1): 195--210.
\url{https://doi.org/10.1139/cjfas-2022-0108}.
\item[2022]
Ward, E.J., L.A.K. Barnett, \textbf{S.C. Anderson}, C.J.C. Commander,
T.E. Essington. Incorporating non-stationary spatial variability into
dynamic species distribution models. ICES Journal of Marine Science.
79(9): 2422-2429. \url{https://doi.org/10.1093/icesjms/fsac179}.
\item[2022]
Essington, T.E. \textbf{S.C. Anderson}, L.A.K. Barnett, H.M. Berger,
S.A. Siedlecki, E.J. Ward. Advancing statistical models to reveal the
effect of dissolved oxygen on the spatial distribution of marine taxa
using thresholds and a physiologically based index. Ecography 2022(8):
e06249. \url{https://doi.org/10.1111/ecog.06249}.
\item[2022]
Thompson, P.L., \textbf{S.C. Anderson}, J. Nephin, D.R. Haggarty, M.
Angelica Peña, P.A. English, K.S.P. Gale, E. Rubidge. Disentangling the
impacts of environmental change and commercial fishing on demersal fish
biodiversity in a northeast Pacific ecosystem. Marine Ecology Progress
Series. 689:137-154. \url{https://doi.org/10.3354/meps14034}.
\item[2022]
Hunsicker, M.E., E.J. Ward, M.A.\ Litzow, \textbf{S.C. Anderson}, C.J.
Harvey, J.C. Field, J. Gao, M.G. Jacox, S. Melin, A.R. Thompson, P.
Warzybok. Tracking and forecasting community responses to climate
perturbations in the California Current Ecosystem. PLOS Climate. 1(3):
e0000014. \url{https://doi.org/10.1371/journal.pclm.0000014}.
\item[2022]
Commander, C.J.C, L.A.K. Barnett, E.J. Ward, \textbf{S.C. Anderson},
T.E. Essington. The shadow model: how and why small choices in spatially
explicit species distribution models affect predictions. PeerJ. 10:
e12783. \url{https://doi.org/10.7717/peerj.12783}.
\item[2022]
Stockdale, J.E., \textbf{S.C. Anderson}, A.M Edwards, S.A. Iyaniwura, N.
Mulberry, M.C. Otterstatter, N.Z. Janjua, D. Coombs, C. Colijn, and M.A.
Irvine. Quantifying transmissibility of SARS-CoV-2 and impact of
intervention within long-term healthcare facilities. Royal Society Open
Science. 9(1): 211710. \url{https://doi.org/10.1098/rsos.211710}.
\item[2022]
Ward, E.J., \textbf{S.C. Anderson}, M.D.\ Hunsicker, and M.A.\ Litzow.
Smoothed dynamic factor analysis for identifying trends in multivariate
time series. Methods in Ecology and Evolution. 13(4):908-918.
\url{https://doi.org/10.1111/2041-210X.13788}.
\item[2021]
English, P., E.J. Ward, C.N. Rooper, R.E. Forrest, L.A. Rogers, K.L.
Hunter, A.M. Edwards, B.M. Connors, \textbf{S.C. Anderson}. Contrasting
climate velocity impacts in warm and cool locations show that effects of
marine warming are worse in already warmer temperate waters. Fish and
Fisheries. 23(1) 239--255. \url{https://doi.org/10.1111/faf.12613}.
{[}\href{https://www.dropbox.com/s/cmkzlxgiez6rzc0/English_etal_2022_contrasting_climate_velocity.pdf?dl=1}{PDF}{]}
\item[2021]
Evans R., P.A. English , \textbf{S.C. Anderson}, S. Gauthier, C.L.K.
Robinson. Factors affecting the seasonal distribution and biomass of
\emph{E. pacifica} and \emph{T. spinifera} along the Pacific coast of
Canada: A spatiotemporal modelling approach. PLOS ONE. 2021 May
14;16(5):e0249818. \url{https://doi.org/10.1371/journal.pone.0249818}.
\item[2021]
Freshwater C., \textbf{S.C. Anderson}, T.D. Beacham, W. Luedke, C. Wor,
J. King. An integrated model of seasonal changes in stock composition
and abundance with an application to Chinook salmon. PeerJ 9:e11163.
\url{https://doi.org/10.7717/peerj.11163}.
\item[2021]
\textbf{Anderson, S.C.}, P.R. Elsen, B.B. Hughes, R.K. Tonietto, M.C.
Bletz, D.A. Gill, M.A.\ Holgerson, S.E. Kuebbing, C. McDonough MacKenzie,
M.H. Meek, D. Veríssimo. Trends in ecology and conservation over eight
decades. Frontiers in Ecology and the Environment. 19(5): 274--282.
\url{http://doi.org/10.1002/fee.2320}.
\item[2021]
\textbf{Anderson, S.C.}, Mulberry, N., Edwards, A.M., Stockdale, J.E.,
Iyaniwura, S.A., Falcao, R.C., Otterstatter, M.C., Janjua, N.Z., Coombs,
D., and Colijn, C. 2020. How much leeway is there to relax COVID-19
control measures? Epidemics. 35: 100453.
\url{https://doi.org/10.1016/j.epidem.2021.100453}.
\item[2021]
Barnett, L.A.K., E.J. Ward, \textbf{S.C. Anderson}. Improving estimates
of species distribution change by incorporating local trends. Ecography.
44: 427--439. \url{https://doi.org/10.1111/ecog.05176}.
\item[2020]
\textbf{Anderson, S.C.}, A.M. Edwards, M. Yerlanov, N. Mulberry, J.
Stockdale, S.A. Iyaniwura, R.C. Falcao, M.C. Otterstatter, M.A.\ Irvine,
N.Z. Janjua, D. Coombs, C. Colijn. Quantifying the impact of COVID-19
control measures using a Bayesian model of physical distancing. PLOS
Computational Biology. 16(12): e1008274.
\url{https://doi.org/10.1371/journal.pcbi.1008274}. medRxiv preprint:
\url{https://doi.org/10.1101/2020.04.17.20070086}.
\item[2020]
Maureaud, A., R. Frelat, L. Pecuchet, N.L. Shackell, B. Merigot, M.L.
Pinsky, K. Amador, \textbf{S.C. Anderson}, \ldots{} (67 co-authors),
J.T. Thorson. Are we ready to track climate-driven shifts in marine
species across international boundaries? A global survey of scientific
bottom trawl data. Global Change Biology. 27: 220--236.
\url{https://doi.org/10.1111/gcb.15404}.
\item[2020]
Cahill, C.L., \textbf{S.C. Anderson}, A.J. Paul. L. Macphearson, M.G.
Sullivan, B. van Poorten, C. Walters, J.R. Post. A spatial-temporal
approach to modeling somatic growth across recreational fisheries
landscapes. Canadian Journal of Fisheries and Aquatic Science.
77(11):1822--1835. \url{https://doi.org/10.1139/cjfas-2019-0434}.
\item[2020]
Litzow, M.A., M.E. Hunsicker, E.J. Ward, \textbf{S.C. Anderson}, J. Gao
, S. Zador, S. Batten , S. Dressel, J. Duffy-Anderson, E. Fergusson, R.
Hopcroft, B.J. Laurel, R. O'Malley Evaluating ecosystem change as Gulf
of Alaska temperature exceeds the limits of preindustrial variability.
Progress in Oceanography. 186:102393.
\url{https://doi.org/10.1016/j.pocean.2020.102393}.
\item[2020]
\textbf{Anderson, S.C.}, E.A. Keppel, A.M. Edwards. Reproducible
visualization of raw fisheries data for 113 species improves
transparency, assessment efficiency, and monitoring. Fisheries.
45:535--543. \url{https://doi.org/10.1002/fsh.10441}.
(\href{https://www.dropbox.com/s/4mmnomvmpg0dbky/Anderson_etal_2020_reproducible_visualization_preprint.pdf?dl=1}{Preprint
PDF})
\item[2020]
Huynh, Q.C., A.R. Hordyk, R.E. Forrest, C.E. Porch, \textbf{S.C.
Anderson}, T.R. Carruthers. The interim management procedure approach
for assessed stocks: Responsive management advice and lower assessment
frequency. Fish and Fisheries. 21(3):663-679.
\url{https://doi.org/10.1111/faf.12453}.
\item[2020]
Free, C.M., O.P. Jensen, \textbf{S.C. Anderson}, N.L. Gutierrez, K.M.
Kleisner, C. Longo, C. Minto, G.C. Osio, J.C. Walsh. Blood from a stone:
performance of catch-only methods in estimating stock biomass status.
Fisheries Research. 223:105452.
\url{https://doi.org/10.1016/j.fishres.2019.105452}.
\item[2020]
Veríssimo, D., \textbf{S.C. Anderson}, M. Tlusty. Did the movie Finding
Dory increase demand for blue tang fish? Ambio. 49:903--911.
\url{https://doi.org/10.1007/s13280-019-01233-7}.
\item[2019]
Freshwater, C., \textbf{S.C. Anderson}, K.R. Holt, A.-M. Huang, C.A.
Holt. Weakened portfolio effects constrain management effectiveness for
population aggregates. Ecological Applications. 29(7):e01966.
\url{https://doi.org/10.1002/eap.1966}.
\item[2019]
Ward, E.J., \textbf{S.C. Anderson}, L.A. Damiano, M.E. Hunsicker,
M.A.\ Litzow. Modeling regimes with extremes: the `bayesdfa' package for
identifying and forecasting common trends and anomalies in multivariate
time-series data. The R Journal. 11(2):46--55.
\url{https://doi.org/10.32614/RJ-2019-007}.
\item[2019]
Afflerbach, J.C., M. Frazier, H.E. Froehlich, \textbf{S.C. Anderson},
B.S. Halpern. Quantifying uncertainty in the wild-caught fisheries goal
of the Ocean Health Index. Fish and Fisheries. 20(2):343--354.
\url{https://doi.org/10.1111/faf.12346}.
\item[2019]
Beaudreau, A.H., E.J. Ward, R.E. Brenner, A.O. Shelton, J.T. Watson,
J.C. Womack, \textbf{S.C. Anderson}, A.C. Haynie, K.N. Marshall, B.C.
Williams. Thirty years of change and the future of Alaskan fisheries:
Shifts in fishing participation and diversification in response to
environmental, regulatory, and economic pressures. Fish and Fisheries.
20(4):601--619. \url{https://doi.org/10.1111/faf.12364}.
\item[2019]
Harvey, B.J., R.A. Andrus, \textbf{S.C. Anderson}. Incorporating
biophysical gradients and uncertainty into burn severity maps in a
temperate fire-prone forested region. Ecosphere. 10(2):02600.
\url{https://doi.org/10.1002/ecs2.2600}.
\item[2019]
Marshall, K.N., J.T. Duffy-Anderson, E.J. Ward, \textbf{S.C. Anderson},
M.E. Hunsicker, B.C. Williams. Long-term trends in ichthyoplankton
assemblage structure, biodiversity, and synchrony in the Gulf of Alaska
and their relationships to climate. Progress in Oceanography.
170:134--145. \url{https://doi.org/10.1016/j.pocean.2018.11.002}.
\item[2019]
\textbf{Anderson, S.C.}, Ward, E. J. Black swans in space: modelling
spatiotemporal processes with extremes. Ecology. 100(1):e02403.
\url{https://doi.org/10.1002/ecy.2403}. glmmfields R package:
\href{https://cran.r-project.org/package=glmmfields}{CRAN},
\href{https://github.com/seananderson/glmmfields}{GitHub}.
\item[2018]
Walsh, J.C., C. Minto, E. Jardim, \textbf{S.C. Anderson}, O.P. Jensen,
J. Afflerbach, K.M. Kleisner, M. Dickey-Collas, C. Longo, G.C. Osio,
E.R. Selig, J.T. Thorson, M.B. Rudd, K.J. Papacostas, J.N. Kittinger,
A.A. Rosenberg, A.B. Cooper. Trade-offs for data-limited fisheries when
using harvest strategies based on catch-only models. Fish and Fisheries.
19(6):1130--1146. \url{https://doi.org/10.1111/faf.12316}.
\item[2018]
Ward, E.J., \textbf{S.C. Anderson}, A.O. Shelton, R.E. Brenner,
M.D.\ Adkison, A.H. Beaudreau, J.T. Watson, J.C. Shriver, A.C. Haynie,
B.C. Williams. Effects of increased specialization on revenue of Alaskan
salmon fishers over four decades. Journal of Applied Ecology.
55(3):1082--1091. \url{https://doi.org/10.1111/1365-2664.13058}.
\item[2018]
Hughes, B., S. Lummis, \textbf{S.C. Anderson}, K. Kroeker. Unexpected
resilience of a seagrass system exposed to multiple stressors. Global
Change Biology. 24(1) 224--234. \url{https://doi.org/10.1111/gcb.13854}.
\item[2018]
Rosenberg, A.A., K.M. Kleisner, J. Afflerbach, \textbf{S.C. Anderson},
M. Dickey-Collas, A.B. Cooper, M.J. Fogarty, E.A. Fulton, N.L.
Gutiérrez, K.J.W. Hyde, E. Jardim, O.P. Jensen, T. Kristiansen, C.
Longo, C.V. Minte-Vera, C. Minto, I. Mosqueira, G.C. Osio, D. Ovando,
E.R. Selig, J.T. Thorson, J.C. Walsh, Y. Ye. Applying a new ensemble
approach to estimating stock status of marine fisheries around the
world. Conservation Letters. 11(1): e12363.
\url{https://doi.org/10.1111/conl.12363}.
\item[2017]
\textbf{Anderson, S.C.}, T.A. Branch, A.B. Cooper, N.K. Dulvy. Reply to
Youngflesh and Lynch: Migration and population growth rate in animal
black-swan events. Proceedings of the National Academy of Sciences.
114(43): E8955--E8956. \url{https://doi.org/10.1073/pnas.1714157114}.
\item[2017]
\textbf{Anderson, S.C.}, E.J. Ward, A.O. Shelton, M.D.\ Adkison, A.H.
Beaudreau, R.E. Brenner, A.C. Haynie, J.C. Shriver, J.T. Watson, B.C.
Williams. Benefits and risks of diversification for individual fishers.
Proceedings of the National Academy of Sciences. 114(40):10797--10802.
\url{https://doi.org/10.1073/pnas.1702506114}.
\item[2017]
Megias, D.A., \textbf{S.C. Anderson}, R.J. Smith, D. Veríssimo.
Investigating the impact of media on demand for wildlife: a case study
of Harry Potter and the UK trade in owls. PLoS ONE. 12(10): e0182368.
\url{https://doi.org/10.1371/journal.pone.0182368}.
\item[2017]
Chezik, K.A., \textbf{S.C. Anderson}, J.W. Moore. River networks dampen
long-term hydrological signals of climate change. Geophysical Research
Letters. 44(14):7256--7264. \url{https://doi.org/10.1002/2017GL074376}.
\item[2017]
\textbf{Anderson, S.C.}, T.A. Branch, A.B. Cooper, N.K. Dulvy.
Black-swan events in animal populations. Proceedings of the National
Academy of Sciences. 114(12): 3252--3257.
\url{https://doi.org/10.1073/pnas.1611525114}.
\item[2017]
\textbf{Anderson, S.C.}, A.B. Cooper, O.P. Jensen, C. Minto, J.T.
Thorson, J.C. Walsh, J. Afflerbach, M. Dickey-Collas, K.M. Kleisner, C.
Longo, G.C. Osio, D. Ovando, I. Mosqueira, A.A. Rosenberg, E.R. Selig.
Improving estimates of population status and trend with superensemble
models. Fish and Fisheries. 18(4): 732--741.
\url{https://doi.org/10.1111/faf.12200}.
\item[2016]
Trebilco, R., N.K. Dulvy, \textbf{S.C. Anderson}, A.K. Salomon. The
paradox of inverted biomass pyramids in kelp forest fish communities.
Proceedings of the Royal Society B: Biological Sciences. 283: 20160816.
\url{https://doi.org/10.1098/rspb.2016.0816}.
\item[2016]
Artelle, K.A., \textbf{S.C. Anderson}, J.D. Reynolds, A.B. Cooper, P.C.
Paquet, C.T. Darimont. Ecology of conflict: marine food supply affects
human-wildlife interactions on land. Scientific Reports. 6: 25936.
\url{https://doi.org/10.1038/srep25936}
(\href{http://www.nature.com/articles/srep25936.pdf}{PDF}).
\item[2016]
Kuriyama, P.T., K. Ono, F. Hurtado-Ferro, A.C. Hicks, I.G. Taylor, R.R.
Licandeo, K.F. Johnson, \textbf{S.C. Anderson}, C.C. Monnahan, M.B.
Rudd, C.C. Stawitz, J.L. Valero. An empirical weight-at-age approach
reduces estimation bias compared to modeling parametric growth in
integrated, statistical stock assessment models when growth is time
varying. Fisheries Research. 180: 119--127.
\url{https://doi.org/10.1016/j.fishres.2015.09.007}
(\href{https://www.dropbox.com/s/425a18a4xm9kq8l/Kuriyama_etal_2016_empirical.pdf?dl=1}{PDF}).
\item[2016]
Monnahan, C.C., K. Ono, \textbf{S.C. Anderson}, M.B. Rudd, A.C. Hicks,
F. Hurtado-Ferro, K.F. Johnson, P.T. Kuriyama, R.R. Licandeo, C.C.
Stawitz, I.G. Taylor, J.L. Valero. The effect of length bin width on
growth estimation in integrated age-structured stock assessments.
Fisheries Research. 180: 103--112.
\url{https://doi.org/10.1016/j.fishres.2015.11.002}
(\href{https://www.dropbox.com/s/uksmdtptby9w0ku/Monnahan_etal_2016_binwidth.pdf?dl=1}{PDF}).
\item[2015]
Boudreau, S.A., \textbf{S.C. Anderson}, B. Worm. Top-down and bottom-up
forces interact at thermal range extremes on American lobster. Journal
of Animal Ecology. 84(3): 840--850.
\url{https://doi.org/10.1111/1365-2656.12322}
(\href{https://www.dropbox.com/s/w027swfx2o8hgvl/Boudreau_etal_2015_lobster.pdf?dl=1}{PDF}).
\item[2015]
\textbf{Anderson, S.C.}, J.W. Moore, M.M. McClure, N.K. Dulvy, A.B.
Cooper. Portfolio conservation of metapopulations under climate change.
Ecological Applications. 25(2): 559--572.
\url{https://doi.org/10.1890/14-0266.1}
(\href{https://www.dropbox.com/s/141rsnv5rc7mi5i/Anderson_etal_2015_salmonportfolios.pdf?dl=1}{PDF}).
\item[2015]
Orzechowski, E.A., R. Lockwood, J.E.K. Byrnes, \textbf{S.C. Anderson},
S. Finnegan, Z.V. Finkel, P.G. Harnik, D.R. Lindberg, L.H. Liow, H.K.
Lotze, C.R. McClain, J.L. McGuire, A. O'Dea, J.M. Pandolfi, C. Simpson,
D.P. Tittensor. Marine extinction risk shaped by trait-environment
interactions over 500 million years. Global Change Biology. 21(10):
3595--3607. \url{https://doi.org/10.1111/gcb.12963}
(\href{https://sean.updog.co/papers/Orzechowski_etal_2015_paleometa.pdf}{PDF}).
\item[2015]
Finnegan, S.*, \textbf{S.C. Anderson}*, P.G. Harnik*, C. Simpson, D.P.
Tittensor, J.E. Byrnes, Z.V. Finkel, D.R. Lindberg, L.H. Liow, R.
Lockwood, H.K. Lotze, C.R. McClain, J.L. McGuire, A. O'Dea, J.M.
Pandolfi. Paleontological baselines for evaluating extinction risk in
the modern oceans. Science. 348(6234): 567--570.
\url{https://doi.org/10.1126/science.aaa6635} (\emph{*Authors
contributed equally}).
\item[2015]
Hurtado-Ferro, F., C.S. Szuwalski, J.L. Valero, \textbf{S.C. Anderson},
C.J. Cunningham, K.F. Johnson, R.R. Licandeo, C.R. McGilliard, C.C.
Monnahan, M.L. Muradian, K. Ono, K.A. Vert-Pre, A.R. Whitten, A.E. Punt.
Looking in the rear-view mirror: bias and retrospective patterns in
integrated, age-structured stock assessment models. ICES Journal of
Marine Sciences. 72(1): 99--110.
\url{https://doi.org/10.1093/icesjms/fsu198}
(\href{https://sean.updog.co/papers/Hurtado-Ferro_etal_2014_retrospective.pdf}{PDF}).
\item[2015]
Johnson, K.F., C.C. Monnahan, C.R. McGilliard, K.A. Vert-pre,
\textbf{S.C. Anderson}, C.J. Cunningham, F. Hurtado-Ferro, R.R.
Licandeo, M.L. Muradian, K. Ono, C.S. Szuwalski, J.L. Valero, A.R.
Whitten, A.E. Punt. Time-varying natural mortality in fisheries stock
assessment models: identifying a default approach. ICES Journal of
Marine Science. 72(1): 137--150.
\url{https://doi.org/10.1093/icesjms/fsu055}
(\href{http://icesjms.oxfordjournals.org/content/early/2014/04/09/icesjms.fsu055.full.pdf?keytype=ref&ijkey=NEXmZIkz3289u3z}{PDF}).
\item[2015]
Ono, K., R. Licandeo, M.L. Muradian, C.J. Cunningham, \textbf{S.C.
Anderson}, F. Hurtado-Ferro, K.F. Johnson, C.R. McGilliard, C.C.
Monnahan, C.S. Szuwalski, J.L. Valero, K.A. Vert-Pre, A.R. Whitten, A.E.
Punt. The importance of length and age composition data in statistical
age-structured models for marine species. ICES Journal of Marine
Science. 72(1): 31--43. \url{https://doi.org/10.1093/icesjms/fsu007}
(\href{http://icesjms.oxfordjournals.org/content/early/2014/02/20/icesjms.fsu007.full.pdf}{PDF}).
\item[2014]
Farmer, R.G., M.L. Leonard, J.E. Mills Flemming, \textbf{S.C. Anderson}.
Observer aging and long-term avian survey data quality. Ecology and
Evolution. 4(12): 2563--2576. \url{https://doi.org/10.1002/ece3.1101}
(\href{http://onlinelibrary.wiley.com/doi/10.1002/ece3.1101/pdf}{PDF}).
\href{http://news.nationalgeographic.com/news/2014/08/140805-aging-birders-breeding-bird-survey-volunteers-science/}{National
Geographic news story}.
\item[2014]
\textbf{Anderson, S.C.}, C.C. Monnahan, K.F. Johnson, K. Ono, J.L.
Valero. ss3sim: An R package for fisheries stock assessment simulation
with Stock Synthesis. PLoS ONE. 9(4): e92725.
\url{https://doi.org/10.1371/journal.pone.0092725}
(\href{http://www.plosone.org/article/fetchObject.action?uri=info\%3Adoi\%2F10.1371\%2Fjournal.pone.0092725&representation=PDF}{PDF}).
(\href{http://cran.r-project.org/web/packages/ss3sim/index.html}{R
package}, \href{https://github.com/ss3sim/ss3sim}{code}).
\item[2014]
O'Regan, S.M., W.J. Palen, \textbf{S.C. Anderson}. Climate warming
mediates negative impacts of rapid pond drying for three amphibian
species. Ecology. 95(4): 845--855.
\url{https://doi.org/10.1890/13-0916.1}
(\href{http://onlinelibrary.wiley.com/doi/10.1890/13-0916.1/epdf}{PDF}).
\href{https://facultyopinions.com/prime/718498042}{F1000 Prime
recommended}.
\item[2013]
Favaro, B., D.C. Braun, \textbf{Earth2Ocean Research Derby}. Research
Derby: A pressure cooker for creative collaborative science. Ideas in
Ecology and Evolution. Ideas in Ecology and Evolution. 6: 40--46.
\url{https://doi.org/10.4033/iee.v6i1.4931} (Open access).
\item[2013]
\textbf{Anderson, S.C.}, A.B. Cooper, N.K. Dulvy. Ecological prophets:
Quantifying metapopulation portfolio effects. Methods in Ecology and
Evolution. 4(10): 971--981.
\url{https://doi.org/10.1111/2041-210X.12093}
(\href{https://www.dropbox.com/s/7tx1h1pkmmp222j/Anderson_etal_2013_ecological_prophets_with_SOM.pdf?dl=1}{PDF},
\href{https://github.com/seananderson/ecofolio}{GitHub R package}).
\item[2013]
Artelle, K.A., \textbf{S.C. Anderson}, A.B. Cooper, P.C. Paquet, J.D.
Reynolds, C.T. Darimont. Confronting uncertainty in wildlife management:
performance of grizzly bear management. PLoS ONE. 8(11): e78041.
\url{https://doi.org/10.1371/journal.pone.0078041}
(\href{http://www.plosone.org/article/fetchObject.action?uri=info\%3Adoi\%2F10.1371\%2Fjournal.pone.0078041&representation=PDF}{PDF}).
\item[2013]
Phillis*, C.C., S.M. O'Regan*, S.J. Green*, J.E. Bruce*, \textbf{S.C.
Anderson}, J.N. Linton, Earth2Ocean Research Derby, B. Favaro. Multiple
pathways to conservation success. Conservation Letters. 6(2): 98--106.
\url{https://doi.org/10.1111/j.1755-263X.2012.00294.x}
(\href{https://www.dropbox.com/s/cxt848ng5x4hc4t/Phillis_etal_2012_Multiple_pathways_to_conservation_success.pdf?dl=1}{PDF},
\href{https://github.com/seananderson/conservation_pathways}{GitHub
repository}). (*Authors contributed equally, listed in reverse
alphabetical order).
\item[2012]
Harnik, P.G., H.K. Lotze, \textbf{S.C. Anderson}, Z.V. Finkel, S.
Finnegan, D.R. Lindberg, L.H. Liow, R. Lockwood, C.R. McClain, J.L.
McGuire, A. O'Dea, J.M. Pandolfi, C. Simpson, D.P. Tittensor.
Extinctions in ancient and modern seas. Trends in Ecology and Evolution.
27(11): 608--617. \url{https://doi.org/10.1016/j.tree.2012.07.010}
(\href{https://www.dropbox.com/s/rdn9685viqr37qy/Harnik_etal_2012_Extinctions_in_ancient_and_modern_seas.pdf?dl=1}{PDF}).
\item[2012]
\textbf{Anderson, S.C.}, T.A. Branch, D. Ricard, H.K. Lotze. Assessing
global marine fishery status with a revised dynamic catch-based method
and stock-assessment reference points. ICES Journal of Marine Science.
69(8): 1491--1500. \url{https://doi.org/10.1093/icesjms/fss105}.
\item[2011]
\textbf{Anderson, S.C.}, R.G. Farmer, F. Ferretti, A.L.S. Houde, J.A.
Hutchings. Correlates of vertebrate extinction risk in Canada.
BioScience. 61(7): 538--549.
\url{https://doi.org/10.1525/bio.2011.61.7.8}
(\href{https://www.dropbox.com/s/bt16dvi3idw3gdx/Anderson_etal_2011_BioScience_with_supplement.pdf?dl=1}{PDF}).
\item[2011]
\textbf{Anderson, S.C.}, J.E. Mills Flemming, R. Watson, H.K. Lotze.
Serial exploitation of global sea cucumber fisheries. Fish and
Fisheries. 12(3): 317--339.
\url{https://doi.org/10.1111/j.1467-2979.2010.00397.x}
(\href{https://www.dropbox.com/s/d8id4zxe9xv6jau/Anderson_etal_2011_seacucumbers_with_supplement.pdf?dl=1}{PDF}).
\href{http://www.sciencemag.org/content/331/6014/129.1.full}{Featured in
Science}.
\item[2011]
Boudreau, S.A., \textbf{S.C. Anderson}, B. Worm. Top-down interactions
and temperature control of snow crab abundance in the northwest Atlantic
Ocean. Marine Ecology Progress Series. 429: 169--183.
\url{https://doi.org/10.3354/meps09081}
(\href{http://www.int-res.com/articles/meps_oa/m429p169.pdf}{PDF}).
\item[2011]
\textbf{Anderson, S.C.}, J.E. Mills Flemming, R. Watson, H.K. Lotze.
Rapid global expansion of invertebrate fisheries: trends, drivers, and
ecosystem effects. PLoS ONE. 6(3): e14735.
\url{https://doi.org/10.1371/journal.pone.0014735}
(\href{http://www.plosone.org/article/fetchObject.action?uri=info\%3Adoi\%2F10.1371\%2Fjournal.pone.0014735&representation=PDF}{PDF}).
\href{https://f1000.com/prime/9542957}{F1000 Prime recommended}.
\item[2008]
\textbf{Anderson, S.C.}, H.K. Lotze, N.L. Shackell. Evaluating the
knowledge base for expanding low-trophic-level fisheries in Atlantic
Canada. Canadian Journal of Fisheries and Aquatic Sciences. 65(12):
2553--2571. \url{https://doi.org/10.1139/F08-156}
(\href{https://www.dropbox.com/s/b0la81jbqitib6u/Anderson_etal_2008_knowledge.pdf?dl=1}{PDF}).
\end{description}

\section{Peer-reviewed government
reports}\label{peer-reviewed-government-reports}

\begin{description}
\tightlist
\item[2023]
Barrett, T.J., J.R. Marentette, R.E. Forrest, S.C. Anderson, C.A. Holt,
D.W. Ings, M.E. Thiess. Technical Considerations for Stock Status and
Limit Reference Points under the Fish Stocks Provisions. Accepted DFO
Can. Sci. Advis. Sec. Res. Doc.
\item[2022]
Grandin, C.J., \textbf{S.C. Anderson}, P.A. English. Arrowtooth Flounder
(\emph{Atheresthes stomias}) Stock Assessment for the West Coast of
British Columbia in 2021. Accepted DFO Can. Sci. Advis. Sec. Res. Doc.
\item[2022]
DFO. A data synopsis for British Columbia groundfish: 2021 data update.
DFO Can. Sci. Advis. Sec. Sci. Resp. 2022/020.
(\href{https://www.dfo-mpo.gc.ca/csas-sccs/Publications/ScR-RS/2022/2022_020-eng.html}{English
version},
\href{https://www.dfo-mpo.gc.ca/csas-sccs/Publications/ScR-RS/2022/2022_020-fra.html}{French
version}).
\item[2021]
DFO. Status Update of Pacifc Cod (\emph{Gadus macrocephalus}) for West
Coast Vancouver Island (Area 3CD), and Hecate Strait and Queen Charlotte
Sound (Area 5ABCD) in 2020. DFO Can. Sci. Advis. Sec. Sci. Resp.
2021/002.
(\href{https://www.dfo-mpo.gc.ca/csas-sccs/Publications/ResDocs-DocRech/2021/2021_007-eng.html}{English
version},
\href{https://www.dfo-mpo.gc.ca/csas-sccs/Publications/ResDocs-DocRech/2021/2021_007-fra.html}{French
version}).
\item[2021]
\textbf{Anderson, S.C.}, R.E. Forrest, Q.C. Huynh, E.A. Keppel. A
management procedure framework for groundfish in British Columbia. DFO
Can. Sci. Advis. Sec. Res. Doc. 2021/007.
(\href{https://www.dfo-mpo.gc.ca/csas-sccs/Publications/ResDocs-DocRech/2021/2021_007-eng.html}{Research
Document},
\href{https://www.dfo-mpo.gc.ca/csas-sccs/Publications/SAR-AS/2021/2021_002-eng.html}{Associated
Science Advisory Report}).
\item[2021]
Haggarty D.R., Q.C. Huynh, R.E. Forrest, \textbf{S.C. Anderson}, M.J.
Bresch, E.A. Keppel. Evaluation of potential rebuilding strategies for
Inside Yelloweye Rockfish (\emph{Sebastes ruberrimus}) in British
Columbia. DFO Can. Sci. Advis. Sec. Res. Doc. 2021/008.
(\href{https://www.dfo-mpo.gc.ca/csas-sccs/Publications/ResDocs-DocRech/2021/2021_008-eng.html}{Research
Document}).
\item[2020]
Forrest, R.E., \textbf{S.C. Anderson}, C.J. Grandin, P.J. Starr.
Assessment of Pacific Cod (\emph{Gadus macrocephalus}) for Hecate Strait
and Queen Charlotte Sound (Area 5ABCD), and West Coast Vancouver Island
(Area 3CD) in 2018. DFO Can. Sci. Advis. Sec. Res. Doc. 2020/070 iv +
204 p.
(\href{https://www.dfo-mpo.gc.ca/csas-sccs/Publications/ResDocs-DocRech/2020/2020_070-eng.html}{English
version},
\href{https://www.dfo-mpo.gc.ca/csas-sccs/Publications/ResDocs-DocRech/2020/2020_070-fra.html}{French
version}).
\item[2019]
\textbf{Anderson, S.C.}, E.A. Keppel, A.M. Edwards, 2019. A reproducible
data synopsis for over 100 species of British Columbia groundfish. DFO
Can. Sci. Advis. Sec. Res. Doc. 2019/041. vii + 321 p.
(\href{http://www.dfo-mpo.gc.ca/csas-sccs/Publications/ResDocs-DocRech/2019/2019_041-eng.html}{English
version},
\href{http://www.dfo-mpo.gc.ca/csas-sccs/Publications/ResDocs-DocRech/2019/2019_041-fra.html}{French
version}).
\end{description}

\section{Technical Reports}\label{technical-reports}

\begin{description}
\tightlist
\item[2023]
ICES. Workshop on Unavoidable Survey Effort Reduction 2 (WKUSER2). ICES
Scientific Report, 5(13).
\url{https://doi.org/10.17895/ices.pub.22086845.v1}. Co-lead author of
chapter ``TOR III. Modeling and simulations: Further develop model
performance evaluation through simulations, use of auxiliary information
to improve survey data products, including appropriate propagation of
uncertainty''
\item[2021]
English, P.A., \textbf{Anderson, S.C.}, and Workman, G.D. A review of
groundfish surveys in 2019. In State of the physical, biological and
selected fishery resources of Pacific Canadian Marine Ecosystems in
2019. Edited by J.L. Boldt, A. Javorski, and P.C. Chandler.
pp.\ 127--133. Can. Tech. Rep.\ Fish. Aquat. Sci. 3434, Sidney, BC.
(\href{https://publications.gc.ca/site/eng/9.900827/publication.html}{English
version}).
\item[2020]
English, P.A., \textbf{Anderson, S.C.}, and Workman, G.D. A review of
groundfish surveys in 2019. In State of the physical, biological and
selected fishery resources of Pacific Canadian Marine Ecosystems in
2019. Edited by J.L. Boldt, A. Javorski, and P.C. Chandler.
pp.\ 102--106. Can. Tech. Rep.\ Fish. Aquat. Sci. 3377, Sidney, BC.
(\href{https://www.dfo-mpo.gc.ca/oceans/publications/soto-rceo/2019/index-eng.html}{English
version}).
\item[2019]
\textbf{Anderson, S.C.}, and Workman, G.D. A review of groundfish
surveys in 2018 and an introduction to the groundfish data synopsis
report. In State of the physical, biological and selected fishery
resources of Pacific Canadian marine ecosystems in 2018. Edited by J.L.
Boldt, J. Leonard, and P.C. Chandler. pp. 90--94. Can. Tech. Rep.\ Fish.
Aquat. Sci. 3314, Sidney, BC.
(\href{https://www.dfo-mpo.gc.ca/oceans/publications/soto-rceo/2018/index-eng.html}{English
version}).
\item[2018]
Edwards, A.M., Duplisea, D.E., Grinnell, M.H., \textbf{Anderson, S.C.},
Grandin, C.J., Ricard, D., Keppel, E.A., Anderson, E.D., Baker, K.D.,
Benoit, H.P., Cleary, J.S., Connors, B.M., Desgagnes, M., English, P.A.,
Fishman, D.F., Freshwater, C., Hedges, K.J., Holt, C.A., Holt, K.R.,
Kronlund, R., Mariscak, A., Obradovich, S.G., Patten, B.A., Rogers, B.,
Rooper, C.N., Simpson, M.R., Surette, T.J., Tallman, R.F., Wheeland,
L.J., Wor, C., and Zhu, X. 2018. Proceedings of the Technical Expertise
in Stock Assessment (TESA) national workshop on ``Tools for transparent,
traceable, and transferable assessments'' 27--30 November 2018 in
Nanaimo, British Columbia. Can. Tech. Rep.\ Fish. Aquat. Sci. 3290,
Nanaimo, BC.
(\href{https://waves-vagues.dfo-mpo.gc.ca/Library/40750152.pdf}{English
version}).
\end{description}

\section{Awards and scholarships}\label{awards-and-scholarships}

\begin{description}
\tightlist
\item[2020--21]
Co-principal Investigator on NOAA Office of Science and Technology
NMFS-DFO grant ``Characterizing range shifts in groundfish species in
response to local climate velocities''. An extension to apply methods
from a similarly titled ACCASP grant below to NOAA waters.
\item[2020--21]
Collaborator on NOAA Office of Science and Technology NMFS-DFO grant
``Groundfish distribution shifts between British Columbia, Canada, and
Southeast Alaska, USA: climate drivers, ecosystem impacts and management
implications''.
\item[2019]
Canadian Journal of Fisheries and Aquatic Sciences Stevenson
Lectureship. \url{https://cdnsciencepub.com/journal/cjfas/about} A
``lectureship conferred upon a young, energetic and creative researcher
at the cutting edge of an aquatic discipline''.
\item[2018--20]
Principal Investigator of a two-year grant from Fisheries and Oceans
Canada's Aquatic Climate Change Adaptation Services Program (ACCASP).
Project title: Characterizing range shifts in British Columbia
groundfish species in response to local climate velocities.
\item[2017]
Canadian Society for Ecology and Evolution (CSEE) Early Career Award.\\
\url{http://csee-scee.ca/2017-early-career-awards/}
\item[2015]
David H. Smith Conservation Postdoctoral Research Fellowship\\
\url{http://conbio.org/mini-sites/smith-fellows/meet-the-fellows/2015-fellows/sean-anderson}
\item[2015]
Simon Fraser University Dean of Graduate Studies Convocation Medal\\
\url{https://www.sfu.ca/dean-gradstudies/blog/year/2015/06/SeanAnderson.html}
\item[2014]
Garfield Weston Foundation / BC Packers Ltd.\ Graduate Fellowship in
Marine Sciences
\item[2014]
Graduate Fellowship (two semesters), Simon Fraser University
\item[2012--13]
Canadian Fulbright Scholar award to the University of Washington
\item[2011]
Canadian \href{http://goo.gl/nA1zE}{Governor General's Academic Gold
Medal} for the top-ranked Master's Natural Sciences and Engineering
thesis at Dalhousie University in 2010
\item[2011--14]
Provost Prize of Distinction, Simon Fraser University
\item[2011--14]
Natural Sciences and Engineering Research Council of Canada Postgraduate
Scholarship (Doctoral)
\item[2007--10]
Faculty Research Grant Scholarship, Dalhousie University
\item[2007--09]
Graduate Studies Scholarship, Dalhousie University
\item[2007]
Environmental Programmes Honour Society Medal, Dalhousie University
\end{description}

\section{Software}\label{software}

\setlist[itemize]{topsep=2pt}

\textbf{sdmTMB}: \url{https://github.com/pbs-assess/sdmTMB}

\begin{itemize}
\tightlist
\item
  An R package for fitting spatiotemporal species distribution GLMMs
  (generalized linear mixed effects models) with TMB (Template Model
  Builder)
\end{itemize}

\textbf{covidseir}: \url{https://github.com/seananderson/covidseir}

\begin{itemize}
\tightlist
\item
  Primary statistical model for COVID-19 forecasts for Canada (Public
  Health Agency of Canada) since August 2020 and British Columbia since
  April 2020
\item
  Fits a Bayesian SEIR (Susceptible, Exposed, Infectious, Recovered)
  model to daily COVID-19 case data
\item
  Can accommodate multiple types of case data at once (e.g., reported
  cases, hospitalizations, ICU admissions) and accounts for delays
  between symptom onset and case appearance.
\item
  Can evaluate epidemic forecasts given changes to population contact
  rates (e.g., levels of social distancing)
\item
  Published: Anderson et al.\ (2020; PLOS Computational Biology)
\end{itemize}

\textbf{glmmfields}: \url{https://cran.r-project.org/package=glmmfields}

\begin{itemize}
\tightlist
\item
  Bayesian generalized linear mixed models (GLMMs) with robust random
  fields for spatiotemporal modeling
\item
  Allows identification of extreme changes in spatial distribution
  (e.g., rapid range contractions)
\item
  Published: Anderson and Ward (2019; Ecology)
\end{itemize}

\textbf{bayesdfa}: \url{https://cran.r-project.org/package=bayesdfa}

\begin{itemize}
\tightlist
\item
  R package to fit Bayesian Dynamic Factor Analysis (DFA) time series
  models (co-developed with E.J. Ward)
\item
  Can identify common trends, extreme events, or regime shifts in
  multiple ecological timeseries
\item
  Published: Ward et al.\ (2019; The R Journal)
\end{itemize}

\textbf{csasdown}: \url{https://github.com/pbs-assess/csasdown}

\begin{itemize}
\tightlist
\item
  R package for reproducible Canadian Science Advisory Secretariat
  Research Documents, Science Responses, and Technical Reports via R
  Markdown
\item
  Enables fully reproducible CSAS reports going directly from databases
  to analysis to written reports to bring CSAS reports in line with
  modern data analysis best practices
\end{itemize}

\textbf{rosettafish}: \url{https://github.com/pbs-assess/rosettafish}

\begin{itemize}
\tightlist
\item
  Enables the seamless translation of fish- and fisheries-related terms
  in R code to enable reproducible and efficient production of bilingual
  CSAS Research Documents and Science Responses
\end{itemize}

\textbf{gfplot, gfdata, gfsynopsis, and gfdlm}:
\url{https://github.com/pbs-assess/}

\begin{itemize}
\tightlist
\item
  R packages to facilitate rapid and reproducible data extraction, model
  fitting, and plotting of BC groundfish data
\item
  gfsynopsis is used to generate an annual Research Document synopsis
  report showing all available fishery and biological data and model
  fits for 113 species
\item
  gfdlm facilitates applying the management strategy evaluation toolkit
  `MSEtool' to BC groundfish
\end{itemize}

\textbf{ss3sim}: \url{https://github.com/ss3sim/ss3sim}

\begin{itemize}
\tightlist
\item
  R package for flexible, rapid, and reproducible fisheries stock
  assessment simulation testing with the widely used Stock Synthesis
  statistical age-structured stock assessment framework
\item
  Published: Anderson et al.\ (2014; PLOS ONE)
\end{itemize}

\section{Selected first-author invited conference
presentations}\label{selected-first-author-invited-conference-presentations}

\begin{description}
\tightlist
\item[2019]
\textbf{Anderson, S.C.}, E.A. Keppel, A.M. Edwards, P.A. English, E.J.
Ward. An automated synopsis of the state of Pacific Canadian groundfish
and climate impacts. North Pacific Marine Science Organization (PICES)
conference. Victoria, BC, Canada. Invited talk to plenary session (to be
given October 2019).
\item[2018]
\textbf{Anderson, S.C.}, Ward, E. J. Black swans in space: modelling
spatiotemporal processes with extremes. International Statistical
Ecology Conference, St.\ Andrews, Scotland.
(\href{https://www.dropbox.com/s/ufce52ocpycftk1/anderson-ward-isec-2018.pdf?dl=1}{slides})
\item[2017]
\textbf{Anderson, S.C.} Data-driven approaches to quantifying population
status and extremes. Early Career Award keynote lecture at the Canadian
Society for Ecology and Evolution (CSEE), Victoria, BC, Canada.
\item[2016]
\textbf{Anderson, S.C.}, T.A. Branch, A.B. Cooper, N.K. Dulvy.
Black-swan events in animal populations. North America Congress for
Conservation Biology, Madison, WI, United States.
\item[2016]
\textbf{Anderson, S.C.}, A.B. Cooper, O.P. Jensen, C. Minto, J.T.
Thorson, J.C. Walsh, M. Dickey-Collas, K.M. Kleisner, C. Longo, G.C.
Osio, D. Ovando, I. Mosqueira, A.A. Rosenberg, E.R. Selig. Improving
estimates of population status and trend with superensemble models.
World Fisheries Conference, Busan, South Korea. (Presented \emph{in
absentia} by J.T. Thorson.)
\item[2015]
\textbf{Anderson, S.C.}, A.B. Cooper, O.P. Jensen, C. Minto, J.T.
Thorson, J.C. Walsh, M. Dickey-Collas, K.M. Kleisner, C. Longo, G.C.
Osio, D. Ovando, I. Mosqueira, A.A. Rosenberg, E.R. Selig. Improving
estimates of population status and trend with superensemble models.
American Fisheries Society Annual Meeting, Portland, OR, United States.
\end{description}

\section{Teaching}\label{teaching}

\begin{description}
\tightlist
\item[2022]
Two workshops on geostatistical modelling with sdmTMB for the NOAA
(National Oceanic and Atmospheric Administration) PSAW (Protected
Species Assessment Workshop) series.
\href{https://pbs-assess.github.io/sdmTMB-teaching/noaa-psaw-2022/}{Slides
and exercises}, Lecture recordings:
\href{https://youtu.be/DIXa7ngVVL0}{Week 1},
\href{https://youtu.be/VxnqgiAAjfk}{Week 2}.
\item[2020]
Two webinars for the national TESA (Technical Expertise in Stock
Assessment) 2019--2020 webinar series: (1) data visualization with
ggplot2 and (2) efficiently iterating with code using the purrr package
\item[2019]
Two-day workshop on Advanced R topics (functional programming,
debugging, profiling, benchmarking, parallel processing, integrating C++
with R). University of British Columbia.
\url{https://github.com/seananderson/adv-r-course}
\item[2019]
Organizer of \href{https://adv-r.hadley.nz/}{Advanced R} book club at
Pacific Biological Station. Organized seven twice-a-month meetings of
DFO Biologists and Research Scientists to work through the book and
clarify exercise material.
\item[2018]
One-day workshop on Bayesian data analysis for ecologists with Stan for
the David H. Smith Conservation Research Fellowship Program.
\url{https://github.com/seananderson/bayes-course}
\item[2018]
Two two-day workshops on advanced generalized linear mixed-effects
modeling in R for the Simon Fraser University Library Research Commons.
\url{https://github.com/seananderson/glmm-course}
\item[2017]
One-and-a-half-day workshop on linear mixed-effects modeling in R and
generalized linear modeling in R for the David H. Smith Conservation
Postdoctoral Research Fellows.
\url{https://github.com/seananderson/glmm-course}
\item[2016]
Two two-day workshops on advanced generalized linear mixed-effects
modeling in R for the Simon Fraser University Library Research Commons.
\url{https://github.com/seananderson/glmm-course}
\item[2014--16]
Developed self-directed lecture and exercises on ggplot2 for FISH 554:
Beautiful graphics in R, School of Aquatic and Fishery Sciences,
University of Washington, Seattle, WA, United States.\\
\url{http://seananderson.ca/ggplot2-FISH554/}
\item[2013--14]
Organizer of Stats Beerz --- a statistical help group attended by
graduate students and postdocs primarily in the Earth to Oceans research
group, but also the wider SFU Biology and Geography Departments, and the
School of Resource and Environmental Management (REM).
\item[2013]
Two-part workshop on data manipulation for Stats Beerz and Earth to
Oceans groups at Simon Fraser University with approximately 25
participants. An introduction to plyr, advanced concepts with plyr and
function debugging, and an introduction to dplyr.\\
\url{https://github.com/seananderson/plyr-statsbeerz}
\item[2013]
Instructor for BISC-888-1: Data Wrangling and Visualization in R, a
graduate-level course at Simon Fraser University, Burnaby, BC, Canada
with approximately 20 participants. Co-developed curriculum and
developed/delivered lectures, exercises, notes, and assignments for
three of six two-hour classes.\\
\url{https://github.com/seananderson/datawranglR} (see classes 03, 04,
05)
\item[2012]
Introduction to ggplot2.
(\href{http://seananderson.ca/courses/12-ggplot2/ggplot2_notes.pdf}{notes},
\href{http://seananderson.ca/courses/12-ggplot2/ggplot2_slides_with_examples.pdf}{slides})
Lecture for FISH 507H: Beautiful Graphics in R, School of Aquatic and
Fishery Sciences, University of Washington, Seattle, WA, United States
\item[2012]
Workshop on the R package plyr.
(\href{http://seananderson.ca/courses/12-plyr/plyr_2012.pdf}{notes},
\href{http://seananderson.ca/courses/12-plyr/plyr_2012_slides.pdf}{slides},
\href{http://seananderson.ca/courses/12-plyr/plyr_2012_examples.html}{examples})
Branch Lab, School of Aquatic and Fishery Sciences, University of
Washington, Seattle, WA, United States
\item[2011]
Multipanel plotting in R with base graphics.
(\href{http://seananderson.ca/courses/11-multipanel/multipanel.pdf}{notes},
\href{http://seananderson.ca/courses/11-multipanel/multipanel-slides.pdf}{slides}).
Earth2Ocean Research Group, Simon Fraser University, Burnaby, BC,
Canada.
\item[2011]
A brief introduction to R.
(\href{http://seananderson.ca/courses/11-rintro/RIntro.pdf}{notes},
\href{http://seananderson.ca/courses/11-rintro/RIntro.R}{workshop
code}). Earth2Ocean Research Group, Simon Fraser University, Burnaby,
BC, Canada.
\item[2011]
Workshop on the R package plyr
(\href{http://seananderson.ca/courses/11-plyr/plyr.pdf}{notes},
\href{http://seananderson.ca/courses/11-plyr/plyr-slides.pdf}{slides}).
Earth2Ocean Research Group, Simon Fraser University, Burnaby, BC,
Canada.
\item[2007--08]
Teaching Assistant, Organismal Biology and Ecology, Dalhousie
University, two semesters (BIOL 1021).
\item[2007--08]
Teaching Assistant, Marine Mammology, Dalhousie University (BIOL 4060).
\end{description}

\section{Working groups and
workshops}\label{working-groups-and-workshops}

\begin{description}
\tightlist
\item[2020--23]
Joint ICES/PICES Working Group on impacts of warming on growth rates and
fisheries yields (WGGRAFY).
\item[2020--22]
TNC (The Nature Conservancy, based in the California office) working
group on climate change impacts on North American west coast fisheries
\item[2020]
Ocean Frontier Institute/Canadian Statistical Sciences Institute
(CANSSI) working group on \emph{Spatial Modelling of Fishery Survey Data
to Improve Stock Assessment}; invited keynote speaker
\item[2019--20]
NOAA (National Oceanic and Atmospheric Administration) FATE (Fisheries
and the Environment) working group \emph{Spatial ecosystem state and
pressure indicators and the scale-dependence of temporal indicator
performance in the California Current and beyond}
\item[2018--19]
European Commission Joint Research Council working group \emph{A new
approach to strengthen evidence-based fisheries management}
investigating approaches to providing fisheries management scientific
advice based on model ensembles.
\item[2018]
Galway-Mayo Institute of Technology (Ireland) working group on ecosystem
tipping points
\item[2016--17]
NOAA (National Oceanic and Atmospheric Administration) FATE (Fisheries
and the Environment) working group \emph{An early-warning index for
abrupt change in northeast Pacific ecosystems}
\item[2015--16]
NCEAS (National Center for Ecological Analysis and Synthesis, Santa
Barbara, CA) \emph{Applying portfolio effects to the Gulf of Alaska
ecosystem: Did multi-scale diversity buffer against the Exxon Valdez oil
spill?} (\href{https://www.nceas.ucsb.edu/featured/marshall}{URL})
\item[2015--16]
Gordon and Betty Moore Foundation funded working group \emph{Applying
data-limited stock status models and developing management guidance for
unassessed fish stocks}.
\item[2011--13]
NESCent (National Evolutionary Synthesis Center, Durham, NC) Working
Group \emph{Determinants of Extinction in Ancient and Modern Seas} led
by Paul Harnik, Seth Finnegan, and Rowan Lockwood.
(\href{http://www.nescent.org/science/awards_summary.php?id=256}{URL})
\item[2010]
Atlantic Halibut Assessment Science Peer Review Meeting, Fisheries and
Oceans, Dartmouth, NS, Canada.
\item[2007--09]
NCEAS (National Center for Ecological Analysis and Synthesis, Santa
Barbara, CA) Distributed Graduate Seminar, in association with the
Working Group \emph{Finding Common Ground in Marine Conservation and
Management} led by Ray Hilborn and Boris Worm.
(\href{http://www.nceas.ucsb.edu/projects/12307}{URL})
\end{description}

\section{Reviewing}\label{reviewing}

Reviewer for Science, Canadian Journal of Fisheries and Aquatic
Sciences, Fish and Fisheries, Fisheries Research, ICES Journal of Marine
Science, Ecological Applications, Conservation Biology, Proceedings B,
Ecology, Marine Policy, Population Ecology, Journal of Applied Ecology,
International Journal of Tropical Biology and Conservation, Biological
Conservation, Diversity and Distributions, Oikos, Journal of
Environmental Management, Endangered Species Research, Aquatic
Conservation, NSERC Discovery Grants
